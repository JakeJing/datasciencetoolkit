% Options for packages loaded elsewhere
\PassOptionsToPackage{unicode}{hyperref}
\PassOptionsToPackage{hyphens}{url}
\PassOptionsToPackage{dvipsnames,svgnames,x11names}{xcolor}
%
\documentclass[
  letterpaper,
  DIV=11,
  numbers=noendperiod]{scrartcl}

\usepackage{amsmath,amssymb}
\usepackage{iftex}
\ifPDFTeX
  \usepackage[T1]{fontenc}
  \usepackage[utf8]{inputenc}
  \usepackage{textcomp} % provide euro and other symbols
\else % if luatex or xetex
  \usepackage{unicode-math}
  \defaultfontfeatures{Scale=MatchLowercase}
  \defaultfontfeatures[\rmfamily]{Ligatures=TeX,Scale=1}
\fi
\usepackage{lmodern}
\ifPDFTeX\else  
    % xetex/luatex font selection
\fi
% Use upquote if available, for straight quotes in verbatim environments
\IfFileExists{upquote.sty}{\usepackage{upquote}}{}
\IfFileExists{microtype.sty}{% use microtype if available
  \usepackage[]{microtype}
  \UseMicrotypeSet[protrusion]{basicmath} % disable protrusion for tt fonts
}{}
\makeatletter
\@ifundefined{KOMAClassName}{% if non-KOMA class
  \IfFileExists{parskip.sty}{%
    \usepackage{parskip}
  }{% else
    \setlength{\parindent}{0pt}
    \setlength{\parskip}{6pt plus 2pt minus 1pt}}
}{% if KOMA class
  \KOMAoptions{parskip=half}}
\makeatother
\usepackage{xcolor}
\setlength{\emergencystretch}{3em} % prevent overfull lines
\setcounter{secnumdepth}{4}
% Make \paragraph and \subparagraph free-standing
\makeatletter
\ifx\paragraph\undefined\else
  \let\oldparagraph\paragraph
  \renewcommand{\paragraph}{
    \@ifstar
      \xxxParagraphStar
      \xxxParagraphNoStar
  }
  \newcommand{\xxxParagraphStar}[1]{\oldparagraph*{#1}\mbox{}}
  \newcommand{\xxxParagraphNoStar}[1]{\oldparagraph{#1}\mbox{}}
\fi
\ifx\subparagraph\undefined\else
  \let\oldsubparagraph\subparagraph
  \renewcommand{\subparagraph}{
    \@ifstar
      \xxxSubParagraphStar
      \xxxSubParagraphNoStar
  }
  \newcommand{\xxxSubParagraphStar}[1]{\oldsubparagraph*{#1}\mbox{}}
  \newcommand{\xxxSubParagraphNoStar}[1]{\oldsubparagraph{#1}\mbox{}}
\fi
\makeatother

\usepackage{color}
\usepackage{fancyvrb}
\newcommand{\VerbBar}{|}
\newcommand{\VERB}{\Verb[commandchars=\\\{\}]}
\DefineVerbatimEnvironment{Highlighting}{Verbatim}{commandchars=\\\{\}}
% Add ',fontsize=\small' for more characters per line
\newenvironment{Shaded}{}{}
\newcommand{\AlertTok}[1]{\textcolor[rgb]{0.58,0.85,0.30}{\textbf{\colorbox[rgb]{0.30,0.12,0.14}{#1}}}}
\newcommand{\AnnotationTok}[1]{\textcolor[rgb]{0.31,0.63,0.31}{#1}}
\newcommand{\AttributeTok}[1]{\textcolor[rgb]{0.65,0.15,0.64}{#1}}
\newcommand{\BaseNTok}[1]{\textcolor[rgb]{0.60,0.41,0.00}{#1}}
\newcommand{\BuiltInTok}[1]{\textcolor[rgb]{0.65,0.15,0.64}{#1}}
\newcommand{\CharTok}[1]{\textcolor[rgb]{0.31,0.63,0.31}{#1}}
\newcommand{\CommentTok}[1]{\textcolor[rgb]{0.63,0.63,0.65}{\textit{#1}}}
\newcommand{\CommentVarTok}[1]{\textcolor[rgb]{0.89,0.34,0.29}{\textit{#1}}}
\newcommand{\ConstantTok}[1]{\textcolor[rgb]{0.60,0.41,0.00}{#1}}
\newcommand{\ControlFlowTok}[1]{\textcolor[rgb]{0.65,0.15,0.64}{#1}}
\newcommand{\DataTypeTok}[1]{\textcolor[rgb]{0.65,0.15,0.64}{#1}}
\newcommand{\DecValTok}[1]{\textcolor[rgb]{0.60,0.41,0.00}{#1}}
\newcommand{\DocumentationTok}[1]{\textcolor[rgb]{0.89,0.34,0.29}{#1}}
\newcommand{\ErrorTok}[1]{\textcolor[rgb]{0.96,0.28,0.28}{\underline{#1}}}
\newcommand{\ExtensionTok}[1]{\textcolor[rgb]{0.25,0.47,0.95}{\textbf{#1}}}
\newcommand{\FloatTok}[1]{\textcolor[rgb]{0.60,0.41,0.00}{#1}}
\newcommand{\FunctionTok}[1]{\textcolor[rgb]{0.25,0.47,0.95}{#1}}
\newcommand{\ImportTok}[1]{\textcolor[rgb]{0.31,0.63,0.31}{#1}}
\newcommand{\InformationTok}[1]{\textcolor[rgb]{0.77,0.36,0.00}{#1}}
\newcommand{\KeywordTok}[1]{\textcolor[rgb]{0.65,0.15,0.64}{#1}}
\newcommand{\NormalTok}[1]{\textcolor[rgb]{0.22,0.23,0.26}{#1}}
\newcommand{\OperatorTok}[1]{\textcolor[rgb]{0.65,0.15,0.64}{#1}}
\newcommand{\OtherTok}[1]{\textcolor[rgb]{0.15,0.68,0.38}{#1}}
\newcommand{\PreprocessorTok}[1]{\textcolor[rgb]{0.65,0.15,0.64}{#1}}
\newcommand{\RegionMarkerTok}[1]{\textcolor[rgb]{0.16,0.50,0.73}{\colorbox[rgb]{0.08,0.19,0.26}{#1}}}
\newcommand{\SpecialCharTok}[1]{\textcolor[rgb]{0.00,0.52,0.74}{#1}}
\newcommand{\SpecialStringTok}[1]{\textcolor[rgb]{0.85,0.27,0.33}{#1}}
\newcommand{\StringTok}[1]{\textcolor[rgb]{0.31,0.63,0.31}{#1}}
\newcommand{\VariableTok}[1]{\textcolor[rgb]{0.89,0.34,0.29}{#1}}
\newcommand{\VerbatimStringTok}[1]{\textcolor[rgb]{0.85,0.27,0.33}{#1}}
\newcommand{\WarningTok}[1]{\textcolor[rgb]{0.85,0.27,0.33}{#1}}

\providecommand{\tightlist}{%
  \setlength{\itemsep}{0pt}\setlength{\parskip}{0pt}}\usepackage{longtable,booktabs,array}
\usepackage{calc} % for calculating minipage widths
% Correct order of tables after \paragraph or \subparagraph
\usepackage{etoolbox}
\makeatletter
\patchcmd\longtable{\par}{\if@noskipsec\mbox{}\fi\par}{}{}
\makeatother
% Allow footnotes in longtable head/foot
\IfFileExists{footnotehyper.sty}{\usepackage{footnotehyper}}{\usepackage{footnote}}
\makesavenoteenv{longtable}
\usepackage{graphicx}
\makeatletter
\def\maxwidth{\ifdim\Gin@nat@width>\linewidth\linewidth\else\Gin@nat@width\fi}
\def\maxheight{\ifdim\Gin@nat@height>\textheight\textheight\else\Gin@nat@height\fi}
\makeatother
% Scale images if necessary, so that they will not overflow the page
% margins by default, and it is still possible to overwrite the defaults
% using explicit options in \includegraphics[width, height, ...]{}
\setkeys{Gin}{width=\maxwidth,height=\maxheight,keepaspectratio}
% Set default figure placement to htbp
\makeatletter
\def\fps@figure{htbp}
\makeatother

\usepackage{setspace}
\usepackage[most]{tcolorbox}

\newtcolorbox{note}{
	colback=gray!10,
	colframe=gray!80!black,
	boxrule=0.5pt,
	arc=2mm,
	left=6pt,
	right=6pt,
	top=6pt,
	bottom=6pt,
	enhanced,
	before upper={\setstretch{1.2}},
}

\newtcolorbox{example}{
	colback=blue!10,
	colframe=blue!80!black,
	boxrule=0.5pt,
	arc=2mm,
	left=6pt,
	right=6pt,
	top=6pt,
	bottom=6pt,
	enhanced,
	before upper={\setstretch{1.2}},
}

% \usepackage[most]{tcolorbox}
% \newtcolorbox{shadednote}{
%   colback=gray!10,
%   colframe=gray!80!black,
%   boxrule=0.5pt,
%   arc=2mm,
%   left=6pt,
%   right=6pt,
%   top=6pt,
%   bottom=6pt,
%   enhanced,
%   before upper=\relax, % Do nothing, let it inherit
% }
% \newtcolorbox{shadednote}{
%   colback=gray!10,
%   colframe=gray!80!black,
%   boxrule=0.5pt,
%   arc=2mm,
%   left=6pt,
%   right=6pt,
%   top=6pt,
%   bottom=6pt,
% }
\usepackage{booktabs}
\usepackage{longtable}
\usepackage{array}
\usepackage{multirow}
\usepackage{wrapfig}
\usepackage{float}
\usepackage{colortbl}
\usepackage{pdflscape}
\usepackage{tabu}
\usepackage{threeparttable}
\usepackage{threeparttablex}
\usepackage[normalem]{ulem}
\usepackage{makecell}
\usepackage{xcolor}
\KOMAoption{captions}{tableheading}
\makeatletter
\@ifpackageloaded{caption}{}{\usepackage{caption}}
\AtBeginDocument{%
\ifdefined\contentsname
  \renewcommand*\contentsname{Table of contents}
\else
  \newcommand\contentsname{Table of contents}
\fi
\ifdefined\listfigurename
  \renewcommand*\listfigurename{List of Figures}
\else
  \newcommand\listfigurename{List of Figures}
\fi
\ifdefined\listtablename
  \renewcommand*\listtablename{List of Tables}
\else
  \newcommand\listtablename{List of Tables}
\fi
\ifdefined\figurename
  \renewcommand*\figurename{Figure}
\else
  \newcommand\figurename{Figure}
\fi
\ifdefined\tablename
  \renewcommand*\tablename{Table}
\else
  \newcommand\tablename{Table}
\fi
}
\@ifpackageloaded{float}{}{\usepackage{float}}
\floatstyle{ruled}
\@ifundefined{c@chapter}{\newfloat{codelisting}{h}{lop}}{\newfloat{codelisting}{h}{lop}[chapter]}
\floatname{codelisting}{Listing}
\newcommand*\listoflistings{\listof{codelisting}{List of Listings}}
\makeatother
\makeatletter
\makeatother
\makeatletter
\@ifpackageloaded{caption}{}{\usepackage{caption}}
\@ifpackageloaded{subcaption}{}{\usepackage{subcaption}}
\makeatother
\makeatletter
\@ifpackageloaded{tcolorbox}{}{\usepackage[skins,breakable]{tcolorbox}}
\makeatother
\makeatletter
\@ifundefined{shadecolor}{\definecolor{shadecolor}{named}{white}}{}
\makeatother
\makeatletter
\@ifundefined{codebgcolor}{\definecolor{codebgcolor}{HTML}{f8f8f8}}{}
\makeatother
\makeatletter
\ifdefined\Shaded\renewenvironment{Shaded}{\begin{tcolorbox}[boxrule=0pt, borderline west={3pt}{0pt}{shadecolor}, frame hidden, enhanced, sharp corners, breakable, colback={codebgcolor}]}{\end{tcolorbox}}\fi
\makeatother

\ifLuaTeX
  \usepackage{selnolig}  % disable illegal ligatures
\fi
\usepackage{bookmark}

\IfFileExists{xurl.sty}{\usepackage{xurl}}{} % add URL line breaks if available
\urlstyle{same} % disable monospaced font for URLs
\hypersetup{
  pdftitle={VisiData: Commands, Tips, and Tricks},
  pdfauthor={Yingqi Jing},
  colorlinks=true,
  linkcolor={blue},
  filecolor={Maroon},
  citecolor={Blue},
  urlcolor={Blue},
  pdfcreator={LaTeX via pandoc}}


\title{VisiData: Commands, Tips, and Tricks}
\author{Yingqi Jing}
\date{July 21, 2025}

\begin{document}
\maketitle

\renewcommand*\contentsname{Contents}
{
\hypersetup{linkcolor=}
\setcounter{tocdepth}{4}
\tableofcontents
}
\listoffigures
\listoftables

\clearpage

\section{Installation}\label{installation}

\begin{Shaded}
\begin{Highlighting}[]
\ExtensionTok{pip3}\NormalTok{ install visidata}
\ExtensionTok{vd} \AttributeTok{{-}{-}version}
\end{Highlighting}
\end{Shaded}

\section{Help}\label{help}

\begin{itemize}
\tightlist
\item
  \texttt{Ctrl+h}: Show help
\item
  \texttt{z+Ctrl+h}: Show all commands
\end{itemize}

\section{Supported Filetypes}\label{supported-filetypes}

\begin{itemize}
\tightlist
\item
  CSV and other delimiter-separated formats
\item
  Excel spreadsheets \emph{(requires
  \texttt{pip3\ install\ xlrd\ openpyxl})}
\item
  Fixed-width files
\item
  SQLite databases
\item
  PostgreSQL \emph{(requires \texttt{pip3\ install\ psycopg2})}
\item
  MySQL \emph{(requires \texttt{pip3\ install\ mysqlclient})}
\item
  HDF5 \emph{(requires \texttt{pip3\ install\ h5py})}
\item
  \texttt{.sas7bdat} \emph{(requires \texttt{pip3\ install\ sas7bdat})}
\item
  \texttt{.xpt} \emph{(requires \texttt{pip3\ install\ xport})}
\item
  \texttt{.sav} \emph{(requires
  \texttt{pip3\ install\ savReaderWriter})}
\item
  \texttt{.dta} \emph{(requires \texttt{pip3\ install\ pandas})}
\end{itemize}

\section{Opening and Quitting Files}\label{opening-and-quitting-files}

\begin{itemize}
\tightlist
\item
  Open a file: \texttt{vd\ file.csv}
\item
  Quit: \texttt{q} or \texttt{gq} (quit all)
\item
  Specify delimiter:
\end{itemize}

\begin{Shaded}
\begin{Highlighting}[]
\ExtensionTok{vd} \AttributeTok{{-}{-}csv{-}delimiter} \StringTok{"|"}\NormalTok{ filename.csv}
\end{Highlighting}
\end{Shaded}

Or in \textbf{yazi} or \textbf{vifm}, navigate to the file and run:

\begin{Shaded}
\begin{Highlighting}[]
\CommentTok{\# yazi}
\ExtensionTok{fish} \AttributeTok{{-}c} \StringTok{"vd {-}{-}csv{-}delimiter \textquotesingle{};\textquotesingle{}; }\VariableTok{$0}\StringTok{"}
\CommentTok{\# vifm}
\BuiltInTok{:}\NormalTok{!vd }\AttributeTok{{-}{-}csv{-}delimiter} \StringTok{";"}\NormalTok{ \%f}
\end{Highlighting}
\end{Shaded}

\section{Sheets}\label{sheets}

VisiData opens each file or operation in a new \textbf{sheet}.

\begin{itemize}
\tightlist
\item
  \texttt{gS}: View all sheets
\item
  Select sheets (\texttt{s}), then join (\texttt{\&}) them: append,
  inner, outer, or diff join
\end{itemize}

\subsection{Adjusting Columns}\label{adjusting-columns}

\begin{itemize}
\tightlist
\item
  Hide a column: \texttt{-}
\item
  Show all hidden columns: \texttt{gv}
\item
  Auto-adjust column width: \texttt{\_}
\end{itemize}

\subsection{Adding Columns and Rows}\label{adding-columns-and-rows}

\begin{itemize}
\tightlist
\item
  \texttt{za}: Append a blank column
\item
  \texttt{i}: Insert new column
\item
  \texttt{\^{}}: Rename column
\item
  \texttt{a}: Add a new row (not editable in frequency table)
\item
  Delete a column: \texttt{Shift+C}, then \texttt{d}
\item
  Edit a cell: \texttt{e}
\end{itemize}

\subsection{Navigation}\label{navigation}

\begin{itemize}
\tightlist
\item
  Go to beginning of column: \texttt{gh}
\item
  Go to end of column: \texttt{gl}
\end{itemize}

\subsection{Overview / Summary}\label{overview-summary}

\begin{itemize}
\tightlist
\item
  \texttt{Shift+I}: Overview (``bird's eye view'') of the data
\end{itemize}

\subsection{Selecting and Deselecting
Rows}\label{selecting-and-deselecting-rows}

\begin{itemize}
\tightlist
\item
  \texttt{,}: Select rows matching the current cell in the current
  column
\item
  \texttt{s}: Select row
\item
  \texttt{gs}: Select all
\item
  \texttt{u}: Deselect current row
\item
  \texttt{gu}: Deselect all
\item
  \texttt{gd}: Delete all selected rows
\item
  Select by pattern:
\item
  Use \texttt{/} to search column
\item
  Use \texttt{g/} to search all columns
\item
  Use regex with \texttt{\textbackslash{}...} to deselect (e.g., rows
  longer than 3 chars)
\item
  From frequency table:

  \begin{itemize}
  \tightlist
  \item
    Select rows (\texttt{s}), press \texttt{g+Enter} to go back
  \item
    Use \texttt{"} to copy
  \item
    Alternative: \texttt{Shift+F} → select rows → \texttt{q} to go back
  \end{itemize}
\end{itemize}

\subsection{Filtering}\label{filtering}

\begin{itemize}
\tightlist
\item
  \texttt{z\textbar{}}: Filter rows using expressions
\item
  Examples:

  \begin{itemize}
  \tightlist
  \item
    \texttt{OPERATOR\ ==\ "BUSINESS"}
  \item
    \texttt{STATE\ !=\ "FL"}
  \item
    \texttt{Height\ \textgreater{}\ 170}
  \end{itemize}
\end{itemize}

\subsection{Deleting}\label{deleting}

\begin{itemize}
\tightlist
\item
  Delete row: \texttt{d}
\item
  Delete column: \texttt{Shift+C} → \texttt{d}
\item
  Delete all selections: \texttt{gd}
\end{itemize}

\subsection{Replace Values}\label{replace-values}

\begin{itemize}
\tightlist
\item
  Select matching rows with \texttt{,}
\item
  \texttt{ge}: Globally edit selected cells in a column
\end{itemize}

\subsection{Sorting}\label{sorting}

\begin{itemize}
\tightlist
\item
  Sort ascending: \texttt{{[}}
\item
  Sort descending: \texttt{{]}}
\end{itemize}

\subsection{Plotting}\label{plotting}

\begin{enumerate}
\def\labelenumi{\arabic{enumi}.}
\tightlist
\item
  Select x-axis column: \texttt{!}
\item
  Select y-axis column: \texttt{!}
\item
  Optional: select color column
\item
  Ensure cursor is on y-axis (numeric)
\item
  Plot: \texttt{.}
\item
  Adjust aspect ratio: \texttt{z\_} (plot width / height)
\item
  Convert to numeric: \texttt{\%} or \texttt{\#}
\end{enumerate}

\subsection{Command Prompt}\label{command-prompt}

\begin{itemize}
\tightlist
\item
  \texttt{Space}: Launch command prompt
\end{itemize}

\subsection{Mnemonics}\label{mnemonics}

\begin{itemize}
\tightlist
\item
  \texttt{g} = ``global'' or ``all'':
\item
  \texttt{/} = search in column
\item
  \texttt{g/} = search in all columns
\item
  \texttt{z} = ``zoom in'' or ``narrow'':
\item
  \texttt{y} = copy column
\item
  \texttt{zy} = copy cell
\end{itemize}




\end{document}
