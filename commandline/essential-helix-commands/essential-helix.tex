% Options for packages loaded elsewhere
\PassOptionsToPackage{unicode}{hyperref}
\PassOptionsToPackage{hyphens}{url}
\PassOptionsToPackage{dvipsnames,svgnames,x11names}{xcolor}
%
\documentclass[
  letterpaper,
  DIV=11,
  numbers=noendperiod]{scrartcl}

\usepackage{amsmath,amssymb}
\usepackage{iftex}
\ifPDFTeX
  \usepackage[T1]{fontenc}
  \usepackage[utf8]{inputenc}
  \usepackage{textcomp} % provide euro and other symbols
\else % if luatex or xetex
  \usepackage{unicode-math}
  \defaultfontfeatures{Scale=MatchLowercase}
  \defaultfontfeatures[\rmfamily]{Ligatures=TeX,Scale=1}
\fi
\usepackage{lmodern}
\ifPDFTeX\else  
    % xetex/luatex font selection
\fi
% Use upquote if available, for straight quotes in verbatim environments
\IfFileExists{upquote.sty}{\usepackage{upquote}}{}
\IfFileExists{microtype.sty}{% use microtype if available
  \usepackage[]{microtype}
  \UseMicrotypeSet[protrusion]{basicmath} % disable protrusion for tt fonts
}{}
\makeatletter
\@ifundefined{KOMAClassName}{% if non-KOMA class
  \IfFileExists{parskip.sty}{%
    \usepackage{parskip}
  }{% else
    \setlength{\parindent}{0pt}
    \setlength{\parskip}{6pt plus 2pt minus 1pt}}
}{% if KOMA class
  \KOMAoptions{parskip=half}}
\makeatother
\usepackage{xcolor}
\setlength{\emergencystretch}{3em} % prevent overfull lines
\setcounter{secnumdepth}{4}
% Make \paragraph and \subparagraph free-standing
\makeatletter
\ifx\paragraph\undefined\else
  \let\oldparagraph\paragraph
  \renewcommand{\paragraph}{
    \@ifstar
      \xxxParagraphStar
      \xxxParagraphNoStar
  }
  \newcommand{\xxxParagraphStar}[1]{\oldparagraph*{#1}\mbox{}}
  \newcommand{\xxxParagraphNoStar}[1]{\oldparagraph{#1}\mbox{}}
\fi
\ifx\subparagraph\undefined\else
  \let\oldsubparagraph\subparagraph
  \renewcommand{\subparagraph}{
    \@ifstar
      \xxxSubParagraphStar
      \xxxSubParagraphNoStar
  }
  \newcommand{\xxxSubParagraphStar}[1]{\oldsubparagraph*{#1}\mbox{}}
  \newcommand{\xxxSubParagraphNoStar}[1]{\oldsubparagraph{#1}\mbox{}}
\fi
\makeatother


\providecommand{\tightlist}{%
  \setlength{\itemsep}{0pt}\setlength{\parskip}{0pt}}\usepackage{longtable,booktabs,array}
\usepackage{calc} % for calculating minipage widths
% Correct order of tables after \paragraph or \subparagraph
\usepackage{etoolbox}
\makeatletter
\patchcmd\longtable{\par}{\if@noskipsec\mbox{}\fi\par}{}{}
\makeatother
% Allow footnotes in longtable head/foot
\IfFileExists{footnotehyper.sty}{\usepackage{footnotehyper}}{\usepackage{footnote}}
\makesavenoteenv{longtable}
\usepackage{graphicx}
\makeatletter
\def\maxwidth{\ifdim\Gin@nat@width>\linewidth\linewidth\else\Gin@nat@width\fi}
\def\maxheight{\ifdim\Gin@nat@height>\textheight\textheight\else\Gin@nat@height\fi}
\makeatother
% Scale images if necessary, so that they will not overflow the page
% margins by default, and it is still possible to overwrite the defaults
% using explicit options in \includegraphics[width, height, ...]{}
\setkeys{Gin}{width=\maxwidth,height=\maxheight,keepaspectratio}
% Set default figure placement to htbp
\makeatletter
\def\fps@figure{htbp}
\makeatother

\usepackage{setspace}
\usepackage[most]{tcolorbox}

\newtcolorbox{note}{
	colback=gray!10,
	colframe=gray!80!black,
	boxrule=0.5pt,
	arc=2mm,
	left=6pt,
	right=6pt,
	top=6pt,
	bottom=6pt,
	enhanced,
	before upper={\setstretch{1.2}},
}

\newtcolorbox{example}{
	colback=blue!10,
	colframe=blue!80!black,
	boxrule=0.5pt,
	arc=2mm,
	left=6pt,
	right=6pt,
	top=6pt,
	bottom=6pt,
	enhanced,
	before upper={\setstretch{1.2}},
}

% \usepackage[most]{tcolorbox}
% \newtcolorbox{shadednote}{
%   colback=gray!10,
%   colframe=gray!80!black,
%   boxrule=0.5pt,
%   arc=2mm,
%   left=6pt,
%   right=6pt,
%   top=6pt,
%   bottom=6pt,
%   enhanced,
%   before upper=\relax, % Do nothing, let it inherit
% }
% \newtcolorbox{shadednote}{
%   colback=gray!10,
%   colframe=gray!80!black,
%   boxrule=0.5pt,
%   arc=2mm,
%   left=6pt,
%   right=6pt,
%   top=6pt,
%   bottom=6pt,
% }
\usepackage{booktabs}
\usepackage{longtable}
\usepackage{array}
\usepackage{multirow}
\usepackage{wrapfig}
\usepackage{float}
\usepackage{colortbl}
\usepackage{pdflscape}
\usepackage{tabu}
\usepackage{threeparttable}
\usepackage{threeparttablex}
\usepackage[normalem]{ulem}
\usepackage{makecell}
\usepackage{xcolor}
\KOMAoption{captions}{tableheading}
\makeatletter
\@ifpackageloaded{caption}{}{\usepackage{caption}}
\AtBeginDocument{%
\ifdefined\contentsname
  \renewcommand*\contentsname{Table of contents}
\else
  \newcommand\contentsname{Table of contents}
\fi
\ifdefined\listfigurename
  \renewcommand*\listfigurename{List of Figures}
\else
  \newcommand\listfigurename{List of Figures}
\fi
\ifdefined\listtablename
  \renewcommand*\listtablename{List of Tables}
\else
  \newcommand\listtablename{List of Tables}
\fi
\ifdefined\figurename
  \renewcommand*\figurename{Figure}
\else
  \newcommand\figurename{Figure}
\fi
\ifdefined\tablename
  \renewcommand*\tablename{Table}
\else
  \newcommand\tablename{Table}
\fi
}
\@ifpackageloaded{float}{}{\usepackage{float}}
\floatstyle{ruled}
\@ifundefined{c@chapter}{\newfloat{codelisting}{h}{lop}}{\newfloat{codelisting}{h}{lop}[chapter]}
\floatname{codelisting}{Listing}
\newcommand*\listoflistings{\listof{codelisting}{List of Listings}}
\makeatother
\makeatletter
\makeatother
\makeatletter
\@ifpackageloaded{caption}{}{\usepackage{caption}}
\@ifpackageloaded{subcaption}{}{\usepackage{subcaption}}
\makeatother
\makeatletter
\@ifpackageloaded{tcolorbox}{}{\usepackage[skins,breakable]{tcolorbox}}
\makeatother
\makeatletter
\@ifundefined{shadecolor}{\definecolor{shadecolor}{named}{white}}{}
\makeatother
\makeatletter
\@ifundefined{codebgcolor}{\definecolor{codebgcolor}{HTML}{f8f8f8}}{}
\makeatother
\makeatletter
\ifdefined\Shaded\renewenvironment{Shaded}{\begin{tcolorbox}[enhanced, sharp corners, boxrule=0pt, frame hidden, breakable, borderline west={3pt}{0pt}{shadecolor}, colback={codebgcolor}]}{\end{tcolorbox}}\fi
\makeatother

\ifLuaTeX
  \usepackage{selnolig}  % disable illegal ligatures
\fi
\usepackage{bookmark}

\IfFileExists{xurl.sty}{\usepackage{xurl}}{} % add URL line breaks if available
\urlstyle{same} % disable monospaced font for URLs
\hypersetup{
  pdftitle={Essential Helix Editor Commands},
  pdfauthor={Yingqi Jing},
  colorlinks=true,
  linkcolor={blue},
  filecolor={Maroon},
  citecolor={Blue},
  urlcolor={Blue},
  pdfcreator={LaTeX via pandoc}}


\title{Essential Helix Editor Commands}
\author{Yingqi Jing}
\date{July 20, 2025}

\begin{document}
\maketitle

\renewcommand*\contentsname{Contents}
{
\hypersetup{linkcolor=}
\setcounter{tocdepth}{4}
\tableofcontents
}
\listoffigures
\listoftables

\clearpage

\textbf{Note:} In Helix, \texttt{Alt} (or \texttt{A}) functions
similarly to \texttt{Cmd} in Vim. Importantly, Helix uses a
\emph{selection-first} model---you must select text before performing
most actions.

\subsection{\texorpdfstring{\textbf{Navigation}}{Navigation}}\label{navigation}

\begin{itemize}
\tightlist
\item
  \texttt{h} / \texttt{j} / \texttt{k} / \texttt{l}: Move left, down,
  up, right
\item
  \texttt{w} / \texttt{W} / \texttt{e} / \texttt{E} / \texttt{b} /
  \texttt{B}: Word-based motion
\item
  \texttt{i} / \texttt{a} / \texttt{I} / \texttt{A}: Insert modes at
  various positions
\item
  \texttt{f} / \texttt{F} / \texttt{t} / \texttt{T}: Jump to characters
  in the current line
\item
  \texttt{gg}: Go to beginning of file
\item
  \texttt{ge}: Go to end of file
\item
  \texttt{gh}: Go to beginning of line
\item
  \texttt{gl}: Go to end of line
\item
  \texttt{u} / \texttt{U}: Undo / redo
\item
  \texttt{gr}: Go to reference
\item
  \texttt{gd}: Go to definition
\end{itemize}

\subsection{\texorpdfstring{\textbf{Deletion \&
Change}}{Deletion \& Change}}\label{deletion-change}

\begin{itemize}
\tightlist
\item
  \texttt{c}: Change character or selection
\item
  \texttt{dd}: Delete line (remapped)
\end{itemize}

\subsection{\texorpdfstring{\textbf{File \& Command
Access}}{File \& Command Access}}\label{file-command-access}

\begin{itemize}
\tightlist
\item
  \texttt{\textless{}Space\textgreater{}\ f}: File picker
\item
  \texttt{\textless{}Space\textgreater{}\ b}: Buffer picker
\item
  \texttt{\textless{}Space\textgreater{}\ ?}: Command palette
\item
  \texttt{ZZ}: Save and close
\item
  \texttt{ZQ}: Quit without saving
\end{itemize}

\subsection{\texorpdfstring{\textbf{Window
Management}}{Window Management}}\label{window-management}

\begin{itemize}
\tightlist
\item
  \texttt{Ctrl-w\ v}: Split window vertically
\item
  \texttt{zz}: Center the current line on screen
\end{itemize}

\subsection{\texorpdfstring{\textbf{Selection}}{Selection}}\label{selection}

\begin{itemize}
\tightlist
\item
  \texttt{v\ 2w}: Select the next two words
\item
  \texttt{x}: Expand selection downward
\item
  \texttt{X}: Expand selection upward
\item
  \texttt{5x}: Select current + 4 lines below
\item
  \texttt{2xv}: Combine with \texttt{jk} to resize selection
\item
  \texttt{mi"}: Select inside quotes
\item
  \texttt{miw}: Select inside word
\item
  \texttt{mip}: Select inside paragraph
\item
  \texttt{\%}: Select entire file
\item
  \texttt{\%s\textless{}pattern\textgreater{}}: Select file and match
  pattern (regex supported); press \texttt{Esc} then \texttt{,} to exit
  multiple cursors
\item
  \texttt{\textless{}Space\textgreater{}\ /}: Grep for word in current
  directory, then \texttt{Ctrl+v} to open matches in vertical split
\end{itemize}

\subsection{\texorpdfstring{\textbf{Multiple
Cursors}}{Multiple Cursors}}\label{multiple-cursors}

\begin{itemize}
\tightlist
\item
  \texttt{C}: Duplicate cursor to the next match/line
\item
  \texttt{,}: Remove the most recent cursor
\item
  \texttt{vj\ C\ C\ C}: Select 2 lines and press \texttt{C} repeatedly
  to create multiple cursors every 2 lines
\item
  \texttt{Ctrl+A} / \texttt{Ctrl+E}: Move all cursors to start/end of
  line
\item
  \texttt{I} / \texttt{A}: Enter insert mode across all cursors
\end{itemize}

\subsection{\texorpdfstring{\textbf{Copy \&
Paste}}{Copy \& Paste}}\label{copy-paste}

\begin{itemize}
\tightlist
\item
  \texttt{x}: Select line
\item
  \texttt{p}: Paste from register
\end{itemize}

\subsection{\texorpdfstring{\textbf{Commands}}{Commands}}\label{commands}

\begin{itemize}
\tightlist
\item
  \texttt{\textbar{}} or \texttt{Cmd+!}: Pipe selection through shell
  command (e.g., \texttt{\textbar{}\ sort} to sort lines)
\end{itemize}

\subsection{\texorpdfstring{\textbf{Definitions \&
Diagnostics}}{Definitions \& Diagnostics}}\label{definitions-diagnostics}

\begin{itemize}
\tightlist
\item
  \texttt{gd}: Go to definition
\item
  \texttt{Ctrl+o}: Go back to previous location
\item
  \texttt{Ctrl+i}: Go forward
\item
  \texttt{\textless{}Space\textgreater{}\ k}: Show function help
\end{itemize}




\end{document}
