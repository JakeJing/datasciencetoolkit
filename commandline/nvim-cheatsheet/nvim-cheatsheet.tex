% Options for packages loaded elsewhere
\PassOptionsToPackage{unicode}{hyperref}
\PassOptionsToPackage{hyphens}{url}
\PassOptionsToPackage{dvipsnames,svgnames,x11names}{xcolor}
%
\documentclass[
  letterpaper,
  DIV=11,
  numbers=noendperiod]{scrartcl}

\usepackage{amsmath,amssymb}
\usepackage{iftex}
\ifPDFTeX
  \usepackage[T1]{fontenc}
  \usepackage[utf8]{inputenc}
  \usepackage{textcomp} % provide euro and other symbols
\else % if luatex or xetex
  \usepackage{unicode-math}
  \defaultfontfeatures{Scale=MatchLowercase}
  \defaultfontfeatures[\rmfamily]{Ligatures=TeX,Scale=1}
\fi
\usepackage{lmodern}
\ifPDFTeX\else  
    % xetex/luatex font selection
\fi
% Use upquote if available, for straight quotes in verbatim environments
\IfFileExists{upquote.sty}{\usepackage{upquote}}{}
\IfFileExists{microtype.sty}{% use microtype if available
  \usepackage[]{microtype}
  \UseMicrotypeSet[protrusion]{basicmath} % disable protrusion for tt fonts
}{}
\makeatletter
\@ifundefined{KOMAClassName}{% if non-KOMA class
  \IfFileExists{parskip.sty}{%
    \usepackage{parskip}
  }{% else
    \setlength{\parindent}{0pt}
    \setlength{\parskip}{6pt plus 2pt minus 1pt}}
}{% if KOMA class
  \KOMAoptions{parskip=half}}
\makeatother
\usepackage{xcolor}
\setlength{\emergencystretch}{3em} % prevent overfull lines
\setcounter{secnumdepth}{4}
% Make \paragraph and \subparagraph free-standing
\makeatletter
\ifx\paragraph\undefined\else
  \let\oldparagraph\paragraph
  \renewcommand{\paragraph}{
    \@ifstar
      \xxxParagraphStar
      \xxxParagraphNoStar
  }
  \newcommand{\xxxParagraphStar}[1]{\oldparagraph*{#1}\mbox{}}
  \newcommand{\xxxParagraphNoStar}[1]{\oldparagraph{#1}\mbox{}}
\fi
\ifx\subparagraph\undefined\else
  \let\oldsubparagraph\subparagraph
  \renewcommand{\subparagraph}{
    \@ifstar
      \xxxSubParagraphStar
      \xxxSubParagraphNoStar
  }
  \newcommand{\xxxSubParagraphStar}[1]{\oldsubparagraph*{#1}\mbox{}}
  \newcommand{\xxxSubParagraphNoStar}[1]{\oldsubparagraph{#1}\mbox{}}
\fi
\makeatother

\usepackage{color}
\usepackage{fancyvrb}
\newcommand{\VerbBar}{|}
\newcommand{\VERB}{\Verb[commandchars=\\\{\}]}
\DefineVerbatimEnvironment{Highlighting}{Verbatim}{commandchars=\\\{\}}
% Add ',fontsize=\small' for more characters per line
\newenvironment{Shaded}{}{}
\newcommand{\AlertTok}[1]{\textcolor[rgb]{0.58,0.85,0.30}{\textbf{\colorbox[rgb]{0.30,0.12,0.14}{#1}}}}
\newcommand{\AnnotationTok}[1]{\textcolor[rgb]{0.31,0.63,0.31}{#1}}
\newcommand{\AttributeTok}[1]{\textcolor[rgb]{0.65,0.15,0.64}{#1}}
\newcommand{\BaseNTok}[1]{\textcolor[rgb]{0.60,0.41,0.00}{#1}}
\newcommand{\BuiltInTok}[1]{\textcolor[rgb]{0.65,0.15,0.64}{#1}}
\newcommand{\CharTok}[1]{\textcolor[rgb]{0.31,0.63,0.31}{#1}}
\newcommand{\CommentTok}[1]{\textcolor[rgb]{0.63,0.63,0.65}{\textit{#1}}}
\newcommand{\CommentVarTok}[1]{\textcolor[rgb]{0.89,0.34,0.29}{\textit{#1}}}
\newcommand{\ConstantTok}[1]{\textcolor[rgb]{0.60,0.41,0.00}{#1}}
\newcommand{\ControlFlowTok}[1]{\textcolor[rgb]{0.65,0.15,0.64}{#1}}
\newcommand{\DataTypeTok}[1]{\textcolor[rgb]{0.65,0.15,0.64}{#1}}
\newcommand{\DecValTok}[1]{\textcolor[rgb]{0.60,0.41,0.00}{#1}}
\newcommand{\DocumentationTok}[1]{\textcolor[rgb]{0.89,0.34,0.29}{#1}}
\newcommand{\ErrorTok}[1]{\textcolor[rgb]{0.96,0.28,0.28}{\underline{#1}}}
\newcommand{\ExtensionTok}[1]{\textcolor[rgb]{0.25,0.47,0.95}{\textbf{#1}}}
\newcommand{\FloatTok}[1]{\textcolor[rgb]{0.60,0.41,0.00}{#1}}
\newcommand{\FunctionTok}[1]{\textcolor[rgb]{0.25,0.47,0.95}{#1}}
\newcommand{\ImportTok}[1]{\textcolor[rgb]{0.31,0.63,0.31}{#1}}
\newcommand{\InformationTok}[1]{\textcolor[rgb]{0.77,0.36,0.00}{#1}}
\newcommand{\KeywordTok}[1]{\textcolor[rgb]{0.65,0.15,0.64}{#1}}
\newcommand{\NormalTok}[1]{\textcolor[rgb]{0.22,0.23,0.26}{#1}}
\newcommand{\OperatorTok}[1]{\textcolor[rgb]{0.65,0.15,0.64}{#1}}
\newcommand{\OtherTok}[1]{\textcolor[rgb]{0.15,0.68,0.38}{#1}}
\newcommand{\PreprocessorTok}[1]{\textcolor[rgb]{0.65,0.15,0.64}{#1}}
\newcommand{\RegionMarkerTok}[1]{\textcolor[rgb]{0.16,0.50,0.73}{\colorbox[rgb]{0.08,0.19,0.26}{#1}}}
\newcommand{\SpecialCharTok}[1]{\textcolor[rgb]{0.00,0.52,0.74}{#1}}
\newcommand{\SpecialStringTok}[1]{\textcolor[rgb]{0.85,0.27,0.33}{#1}}
\newcommand{\StringTok}[1]{\textcolor[rgb]{0.31,0.63,0.31}{#1}}
\newcommand{\VariableTok}[1]{\textcolor[rgb]{0.89,0.34,0.29}{#1}}
\newcommand{\VerbatimStringTok}[1]{\textcolor[rgb]{0.85,0.27,0.33}{#1}}
\newcommand{\WarningTok}[1]{\textcolor[rgb]{0.85,0.27,0.33}{#1}}

\providecommand{\tightlist}{%
  \setlength{\itemsep}{0pt}\setlength{\parskip}{0pt}}\usepackage{longtable,booktabs,array}
\usepackage{calc} % for calculating minipage widths
% Correct order of tables after \paragraph or \subparagraph
\usepackage{etoolbox}
\makeatletter
\patchcmd\longtable{\par}{\if@noskipsec\mbox{}\fi\par}{}{}
\makeatother
% Allow footnotes in longtable head/foot
\IfFileExists{footnotehyper.sty}{\usepackage{footnotehyper}}{\usepackage{footnote}}
\makesavenoteenv{longtable}
\usepackage{graphicx}
\makeatletter
\def\maxwidth{\ifdim\Gin@nat@width>\linewidth\linewidth\else\Gin@nat@width\fi}
\def\maxheight{\ifdim\Gin@nat@height>\textheight\textheight\else\Gin@nat@height\fi}
\makeatother
% Scale images if necessary, so that they will not overflow the page
% margins by default, and it is still possible to overwrite the defaults
% using explicit options in \includegraphics[width, height, ...]{}
\setkeys{Gin}{width=\maxwidth,height=\maxheight,keepaspectratio}
% Set default figure placement to htbp
\makeatletter
\def\fps@figure{htbp}
\makeatother

\usepackage{setspace}
\usepackage[most]{tcolorbox}

\newtcolorbox{note}{
	colback=gray!10,
	colframe=gray!80!black,
	boxrule=0.5pt,
	arc=2mm,
	left=6pt,
	right=6pt,
	top=6pt,
	bottom=6pt,
	enhanced,
	before upper={\setstretch{1.2}},
}

\newtcolorbox{example}{
	colback=blue!10,
	colframe=blue!80!black,
	boxrule=0.5pt,
	arc=2mm,
	left=6pt,
	right=6pt,
	top=6pt,
	bottom=6pt,
	enhanced,
	before upper={\setstretch{1.2}},
}

% \usepackage[most]{tcolorbox}
% \newtcolorbox{shadednote}{
%   colback=gray!10,
%   colframe=gray!80!black,
%   boxrule=0.5pt,
%   arc=2mm,
%   left=6pt,
%   right=6pt,
%   top=6pt,
%   bottom=6pt,
%   enhanced,
%   before upper=\relax, % Do nothing, let it inherit
% }
% \newtcolorbox{shadednote}{
%   colback=gray!10,
%   colframe=gray!80!black,
%   boxrule=0.5pt,
%   arc=2mm,
%   left=6pt,
%   right=6pt,
%   top=6pt,
%   bottom=6pt,
% }
\usepackage{booktabs}
\usepackage{longtable}
\usepackage{array}
\usepackage{multirow}
\usepackage{wrapfig}
\usepackage{float}
\usepackage{colortbl}
\usepackage{pdflscape}
\usepackage{tabu}
\usepackage{threeparttable}
\usepackage{threeparttablex}
\usepackage[normalem]{ulem}
\usepackage{makecell}
\usepackage{xcolor}
\KOMAoption{captions}{tableheading}
\makeatletter
\@ifpackageloaded{caption}{}{\usepackage{caption}}
\AtBeginDocument{%
\ifdefined\contentsname
  \renewcommand*\contentsname{Table of contents}
\else
  \newcommand\contentsname{Table of contents}
\fi
\ifdefined\listfigurename
  \renewcommand*\listfigurename{List of Figures}
\else
  \newcommand\listfigurename{List of Figures}
\fi
\ifdefined\listtablename
  \renewcommand*\listtablename{List of Tables}
\else
  \newcommand\listtablename{List of Tables}
\fi
\ifdefined\figurename
  \renewcommand*\figurename{Figure}
\else
  \newcommand\figurename{Figure}
\fi
\ifdefined\tablename
  \renewcommand*\tablename{Table}
\else
  \newcommand\tablename{Table}
\fi
}
\@ifpackageloaded{float}{}{\usepackage{float}}
\floatstyle{ruled}
\@ifundefined{c@chapter}{\newfloat{codelisting}{h}{lop}}{\newfloat{codelisting}{h}{lop}[chapter]}
\floatname{codelisting}{Listing}
\newcommand*\listoflistings{\listof{codelisting}{List of Listings}}
\makeatother
\makeatletter
\makeatother
\makeatletter
\@ifpackageloaded{caption}{}{\usepackage{caption}}
\@ifpackageloaded{subcaption}{}{\usepackage{subcaption}}
\makeatother
\makeatletter
\@ifpackageloaded{tcolorbox}{}{\usepackage[skins,breakable]{tcolorbox}}
\makeatother
\makeatletter
\@ifundefined{shadecolor}{\definecolor{shadecolor}{named}{white}}{}
\makeatother
\makeatletter
\@ifundefined{codebgcolor}{\definecolor{codebgcolor}{HTML}{f8f8f8}}{}
\makeatother
\makeatletter
\ifdefined\Shaded\renewenvironment{Shaded}{\begin{tcolorbox}[enhanced, sharp corners, frame hidden, boxrule=0pt, borderline west={3pt}{0pt}{shadecolor}, breakable, colback={codebgcolor}]}{\end{tcolorbox}}\fi
\makeatother

\ifLuaTeX
  \usepackage{selnolig}  % disable illegal ligatures
\fi
\usepackage{bookmark}

\IfFileExists{xurl.sty}{\usepackage{xurl}}{} % add URL line breaks if available
\urlstyle{same} % disable monospaced font for URLs
\hypersetup{
  pdftitle={Neovim Cheat Sheet: Essential Tricks and Shortcuts},
  pdfauthor={Yingqi Jing},
  colorlinks=true,
  linkcolor={blue},
  filecolor={Maroon},
  citecolor={Blue},
  urlcolor={Blue},
  pdfcreator={LaTeX via pandoc}}


\title{Neovim Cheat Sheet: Essential Tricks and Shortcuts}
\author{Yingqi Jing}
\date{July 21, 2025}

\begin{document}
\maketitle

\renewcommand*\contentsname{Contents}
{
\hypersetup{linkcolor=}
\setcounter{tocdepth}{4}
\tableofcontents
}
\listoffigures
\listoftables

\clearpage

\subsection{Installation}\label{installation}

\begin{itemize}
\tightlist
\item
  Clone the dotfiles and copy the Neovim config into your
  \texttt{\textasciitilde{}/.config} directory:
\end{itemize}

\begin{Shaded}
\begin{Highlighting}[]
\FunctionTok{git}\NormalTok{ clone https://github.com/JakeJing/dotfiles.git}
\FunctionTok{mv}\NormalTok{ dotfiles/.config/nvim }\AttributeTok{{-}P}\NormalTok{ \textasciitilde{}/.config/}
\end{Highlighting}
\end{Shaded}

\begin{itemize}
\tightlist
\item
  Open \texttt{plugins.lua} and type \texttt{:w} to launch the
  auto-installation.
\item
  Use \texttt{:checkhealth} to verify that all dependencies are properly
  set up.
\end{itemize}

\subsection{Navigation \& Movements}\label{navigation-movements}

\begin{itemize}
\tightlist
\item
  \texttt{hjkl}: move left, down, up, right
\item
  \texttt{Ctrl+d/u}: scroll down/up one page
\item
  \texttt{Ctrl+n/p}: next/previous file in window
\item
  \texttt{i/I}: insert before cursor / at line start
\item
  \texttt{a/A}: append after cursor / at line end
\item
  \texttt{0} / \texttt{\^{}}: go to line start
\item
  \texttt{\$}: go to line end
\item
  \texttt{o/O}: new line below/above
\item
  \texttt{\%}: jump between matching brackets
\item
  \texttt{b/e/w}: word navigation (back, end, next)
\item
  \texttt{f\{x\}} / \texttt{F\{x\}}: move to char \texttt{\{x\}}
  forward/back
\item
  \texttt{t\{x\}} / \texttt{T\{x\}}: move before char \texttt{\{x\}}
  forward/back
\item
  \texttt{\textquotesingle{}\textquotesingle{}} or \texttt{g:}: go back
  to previous position
\item
  \texttt{o} (in visual mode): toggle selection endpoint
\end{itemize}

\subsection{Bookmarks (with Telescope)}\label{bookmarks-with-telescope}

\begin{itemize}
\tightlist
\item
  \texttt{mm}: toggle bookmark
\item
  \texttt{Shift+n}: next bookmark
\item
  \texttt{Shift+b}: previous bookmark
\item
  \texttt{ma}: view all bookmarks
\item
  \texttt{:Telescope\ vim\_bookmarks\ all}: list all bookmarks
\item
  \texttt{:Telescope\ vim\_bookmarks\ current\_file}: list for current
  file
\end{itemize}

\subsection{Deleting Text}\label{deleting-text}

\begin{itemize}
\tightlist
\item
  \texttt{dd}: delete line
\item
  \texttt{3dd}: delete 3 lines
\item
  \texttt{D} or \texttt{d\$}: delete to line end
\item
  \texttt{d0}: delete to line start
\item
  \texttt{x}: delete character under cursor
\item
  \texttt{dw}, \texttt{diw}, \texttt{daw}: delete word (various scopes)
\item
  \texttt{dip}: delete paragraph
\item
  \texttt{di"}: delete inside quotes
\item
  \texttt{:\%d}: delete entire file content
\item
  \texttt{ci"} / \texttt{ci(}: delete and insert inside quotes/brackets
\end{itemize}

\subsection{Joining Lines}\label{joining-lines}

\begin{itemize}
\tightlist
\item
  \texttt{J}: join lines with space
\item
  \texttt{gJ}: join lines without space
\end{itemize}

\subsection{Increment Numbers}\label{increment-numbers}

\begin{itemize}
\tightlist
\item
  Select numbers with \texttt{Ctrl+v}, then \texttt{g\ Ctrl+a} to
  increment
\end{itemize}

\subsection{Word Count}\label{word-count}

\begin{itemize}
\tightlist
\item
  \texttt{:WordCount}: custom word count (requires user-defined
  function)
\item
  \texttt{\textless{}leader\textgreater{}wc}: shortcut for word count
\end{itemize}

\subsection{Selection (Visual Mode)}\label{selection-visual-mode}

\begin{itemize}
\tightlist
\item
  \texttt{v}, \texttt{V}, \texttt{Ctrl+v}: character, line, and block
  selection
\item
  \texttt{viw}, \texttt{vaw}: inside/around word
\item
  \texttt{vi\{}: inside block (like function)
\item
  \texttt{f\{char\}} in visual mode: fast selection
\end{itemize}

\subsection{Multi-line Insert \& Append}\label{multi-line-insert-append}

\begin{quote}
Note: Only works in \textbf{visual block mode} (\texttt{Ctrl+v}).
\end{quote}

\begin{itemize}
\tightlist
\item
  \textbf{Insert start}: \texttt{Ctrl+v}, select, \texttt{Shift+i},
  type, then \texttt{Esc}
\item
  \textbf{Append end}: \texttt{Ctrl+v}, select, \texttt{Shift+a}, type,
  then \texttt{Esc}
\item
  \textbf{Change text}: select block, press \texttt{c}, type, then
  \texttt{Esc}
\end{itemize}

\subsection{Editing Text}\label{editing-text}

\begin{itemize}
\tightlist
\item
  \texttt{ci(}: change inside parentheses
\item
  \texttt{cip}: change inside paragraph
\item
  \texttt{cw}: change word
\end{itemize}

\subsection{Yank \& Paste}\label{yank-paste}

\begin{itemize}
\item
  \texttt{yy}, \texttt{yiw}, \texttt{ya(}, \texttt{y2w}: yank lines,
  words, brackets
\item
  Use \href{https://github.com/AckslD/nvim-neoclip.lua}{neoclip} for
  yank history:

  \begin{itemize}
  \tightlist
  \item
    \texttt{\textless{}C-c\textgreater{}} to yank
  \item
    \texttt{\textless{}C-p\textgreater{}} to paste from history
  \end{itemize}
\end{itemize}

\subsection{Substitution}\label{substitution}

\begin{itemize}
\item
  \texttt{S}: start substitution (\texttt{:\%s//g} via remap)
\item
  \texttt{cgn}: change next match, repeat with \texttt{.}
\item
  Regex substitution:

\begin{Shaded}
\begin{Highlighting}[]
\NormalTok{:\%s/[{-}.+/a{-}zA{-}Z0{-}9"$]*\textbackslash{}ze:/\textasciigrave{}\textbackslash{}0\textasciigrave{}/g}
\end{Highlighting}
\end{Shaded}
\item
  \texttt{R}: overwrite text
\item
  \texttt{:\textquotesingle{}\textless{},\textquotesingle{}\textgreater{}s/old/new/g}:
  substitute in selection
\item
  \texttt{:cdo\ s/old/new/g}: substitute across quickfix list
\end{itemize}

\subsection{Repeating \& Undoing}\label{repeating-undoing}

\begin{itemize}
\tightlist
\item
  \texttt{.}: repeat last change
\item
  \texttt{u} / \texttt{U}: undo / undo line
\item
  \texttt{Ctrl+r}: redo
\item
  \texttt{Ctrl+s}: open terminal in vertical split
\item
  \texttt{:vs\ \textbar{}\ :term}: open vertical terminal
\end{itemize}

\subsection{Quit \& Save}\label{quit-save}

\begin{itemize}
\tightlist
\item
  \texttt{ZZ}: save and quit
\item
  \texttt{ZQ}: quit without saving
\end{itemize}

\subsection{Key Mapping \& Help}\label{key-mapping-help}

\subsubsection{Special Keys}\label{special-keys}

\begin{Shaded}
\begin{Highlighting}[]
\NormalTok{\textless{}Tab\textgreater{}, \textless{}CR\textgreater{}, \textless{}Esc\textgreater{}, \textless{}Space\textgreater{}, \textless{}A{-}j\textgreater{}, \textless{}C{-}s\textgreater{}, \textless{}Up\textgreater{}, \textless{}F1\textgreater{}...\textless{}F12\textgreater{}, etc.}
\end{Highlighting}
\end{Shaded}

\subsubsection{Check Mappings}\label{check-mappings}

\begin{Shaded}
\begin{Highlighting}[]
\NormalTok{:imap \textless{}A{-}j\textgreater{}}
\NormalTok{:verbose imap \textless{}Tab\textgreater{}}
\end{Highlighting}
\end{Shaded}

\subsubsection{Help}\label{help}

\begin{itemize}
\tightlist
\item
  \texttt{:h\ \{key\}} or \texttt{:help\ ctrl-w\_\textless{}}
\item
  \texttt{:h\ telescope.command}, \texttt{:h\ regex}
\item
  \texttt{fh} or \texttt{\textless{}leader\textgreater{}fh}: floating
  help
\end{itemize}

\subsection{Terminal Tricks}\label{terminal-tricks}

\begin{itemize}
\tightlist
\item
  \texttt{Ctrl+s}: toggle floating terminal
\item
  \texttt{:r\ !ls}: insert output of shell command
\end{itemize}

\subsection{TermVifm + Zoxide}\label{termvifm-zoxide}

\begin{itemize}
\item
  Bind \texttt{vf} to launch \texttt{vifm}
\item
  Use \texttt{zoxide} with \texttt{\textless{}leader\textgreater{}Z} to
  jump to dirs

  \begin{itemize}
  \item
    Install with \texttt{brew\ install\ zoxide}
  \item
    Add to \texttt{config.fish}:

\begin{Shaded}
\begin{Highlighting}[]
\NormalTok{zoxide init fish | source}
\end{Highlighting}
\end{Shaded}
  \end{itemize}
\end{itemize}

\subsection{Search}\label{search}

\begin{itemize}
\tightlist
\item
  \texttt{ff}: find file
\item
  \texttt{fa}: find all in buffer
\item
  \texttt{fw}: find word
\item
  \texttt{\textquotesingle{}Search}: exact match
\item
  \texttt{:set\ hlsearch} / \texttt{:nohlsearch}: toggle highlights
\end{itemize}

\subsection{Buffers}\label{buffers}

\begin{itemize}
\tightlist
\item
  \texttt{:bfirst}, \texttt{:blast}, \texttt{:bnext},
  \texttt{:bprevious}
\item
  \texttt{:ls}: list buffers
\item
  \texttt{:bd\ {[}num{]}}: close buffer
\item
  \texttt{:b\ \textless{}TAB\textgreater{}}: autocomplete open buffers
\item
  \texttt{Ctrl+\^{}}: toggle last buffer
\end{itemize}

\subsection{Window Management}\label{window-management}

\begin{itemize}
\tightlist
\item
  \texttt{Ctrl+w\ w}: switch windows
\item
  \texttt{Ctrl+w\ h/l}: left/right
\item
  \texttt{Ctrl+w\ \_}: maximize
\item
  \texttt{Ctrl+w\ =}: equalize
\item
  \texttt{Ctrl+w\ R}: reverse splits
\item
  \texttt{Ctrl+w\ t\ Ctrl+w\ K}: horizontal → vertical
\item
  \texttt{Ctrl+w\ t\ Ctrl+w\ H}: vertical → horizontal
\item
  \texttt{Ctrl+Cmd+F} or \texttt{fn+F}: full screen
\end{itemize}

\subsection{Git with LazyGit}\label{git-with-lazygit}

\begin{itemize}
\tightlist
\item
  \texttt{Ctrl+g}: open LazyGit
\item
  \texttt{c}: commit
\item
  \texttt{Shift+p}: push
\item
  \texttt{?}: help
\item
  \texttt{gl}: git logs
\end{itemize}

\subsection{Linting \& Formatting}\label{linting-formatting}

\begin{itemize}
\tightlist
\item
  \texttt{:NullLsInfo}: see current null-ls status
\item
  \texttt{:echo\ executable("eslint")}: check if installed
\item
  \texttt{:LspStop}: stop diagnostics
\end{itemize}

\subsection{Registers \& Special
Characters}\label{registers-special-characters}

\begin{Shaded}
\begin{Highlighting}[]
\NormalTok{"\%": current file}
\NormalTok{"\#": alternate file}
\NormalTok{"*", "+": system clipboard}
\NormalTok{"/": last search}
\NormalTok{":": last command}
\NormalTok{"{-}": last small delete}
\NormalTok{".": last insert}
\NormalTok{"=": expression register}
\end{Highlighting}
\end{Shaded}

\subsection{Cmdline Modes}\label{cmdline-modes}

\begin{Shaded}
\begin{Highlighting}[]
\NormalTok{:   Normal command}
\NormalTok{\textgreater{}   Debug mode}
\NormalTok{/   Forward search}
\NormalTok{?   Backward search}
\NormalTok{=   Expression}
\NormalTok{@   Input()}
\NormalTok{{-}   Insert/append text}
\end{Highlighting}
\end{Shaded}

\begin{itemize}
\tightlist
\item
  \texttt{q:}: show command history
\end{itemize}




\end{document}
